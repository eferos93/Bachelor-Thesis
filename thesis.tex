%% Le lingue utilizzate, che verranno passate come opzioni al pacchetto babel. Come sempre, l'ultima indicata sar� quella primaria.
%% Se si utilizzano una o pi� lingue diverse da "italian" o "english", leggere le istruzioni in fondo.
\def\thudbabelopt{english}
%% Valori ammessi per target: bach (tesi triennale), mst (tesi magistrale), phd (tesi di dottorato).
\documentclass[target=bach]{thud}[2019/10/17]

%% --- Informazioni sulla tesi ---
%% Per tutti i tipi di tesi
\title{Design and implementation of a Language Server for the Jolie programming language​ }
\author{Eros Fabrici}
\course{Tecnologie Web e Multimediali}
\supervisor{Prof.\ Marino Miculan}
%% Altri campi disponibili: \reviewer, \tutor, \chair, \date (anno accademico, calcolato in automatico).
%% Con \supervisor, \cosupervisor, \reviewer e \tutor si possono indicare pi� nomi separati da \and.
%% Per le sole tesi di dottorato
%% Campi obbligatori: \title, \author e \course.

%% --- Pacchetti consigliati ---
%% hyperref: Regola le impostazioni della creazione del PDF... più tante altre cose.
%% tocbibind: Inserisce nell'indice anche la lista delle figure, la bibliografia, ecc.

%% --- Stili di pagina disponibili (comando \pagestyle) ---
%% sfbig (predefinito): Apertura delle parti e dei capitoli col numero grande; titoli delle parti e dei capitoli e intestazioni di pagina in sans serif.
%% big: Come "sfbig", solo serif.
%% plain: Apertura delle parti e dei capitoli tradizionali di LaTeX; intestazioni di pagina come "big".

\begin{document}

%% Il frontespizio prima di tutto!
\maketitle

%% Dedica (opzionale)
\begin{dedication}Al mio cane,\par per avermi ascoltato mentre ripassavo le lezioni.\end{dedication}

%% Ringraziamenti (opzionali)
\acknowledgements
Sed vel lorem a arcu faucibus aliquet eu semper tortor. Aliquam dolor lacus, semper vitae ligula sed, blandit iaculis leo. Nam pharetra lobortis leo nec auctor. Pellentesque habitant morbi tristique senectus et netus et malesuada fames ac turpis egestas. Fusce ac risus pulvinar, congue eros non, interdum metus. Mauris tincidunt neque et aliquam imperdiet. Aenean ac tellus id nibh pellentesque pulvinar ut eu lacus. Proin tempor facilisis tortor, et hendrerit purus commodo laoreet. Quisque sed augue id ligula consectetur adipiscing. Vestibulum libero metus, lacinia ac vestibulum eu, varius non arcu. Nam et gravida velit.

%% Sommario (opzionale)
\abstract
Nunc ac dignissim ipsum, quis pulvinar elit. Mauris congue nec leo ornare lobortis. Nulla hendrerit pretium diam nec lobortis. Nullam aliquam laoreet nisl, sit amet facilisis lectus accumsan ut. Duis et elit hendrerit metus venenatis condimentum. Integer id eros molestie, interdum leo sit amet, aliquet metus. Integer fermentum tristique magna, vel luctus neque rhoncus vel. Ut hendrerit et quam et semper. Mauris egestas, odio sed aliquet luctus, magna orci euismod odio, vitae lacinia tellus tellus non lectus. Aliquam urna neque, porta et mattis aliquam, congue sit amet lorem. In ultrices augue sit amet ante vehicula, vitae rhoncus turpis auctor. Donec porta scelerisque eros, at mollis enim imperdiet ut. 

%% Indice
\tableofcontents

%% Lista delle tabelle (se presenti)
%\listoftables

%% Lista delle figure (se presenti)
%\listoffigures

%% Corpo principale del documento
\mainmatter

%% Parte
%% La suddivisione in parti � opzionale; talvolta sono sufficienti i capitoli.
\part{Parte}

%% Capitolo
\chapter{Capitolo}
In hac habitasse platea dictumst. Vestibulum consectetur dictum pellentesque. Suspendisse nunc neque, commodo ac imperdiet nec, sollicitudin vitae libero. Donec bibendum vel nunc vitae pharetra. In vel volutpat odio, et interdum dui. Duis mauris ligula, congue eget molestie at, tincidunt nec diam. Nam vitae eros nec arcu suscipit vehicula. Aliquam consectetur imperdiet elit, eget pretium arcu fringilla at. Maecenas \cite{Knu86} sed libero pulvinar, mattis tortor vel, fermentum enim.

%% Sezione
\section{Sezione}
Donec pulvinar neque non lectus vulputate pellentesque. Quisque rutrum arcu velit, in feugiat sapien posuere vel. Praesent metus orci, aliquam ac cursus eget, fermentum a nisl. Etiam eu augue lacus. Nam nisi sapien, mattis sed vehicula non, pellentesque at quam. Sed euismod, dolor nec commodo lobortis, erat erat ultricies eros, bibendum dictum nulla felis in dui. Nulla blandit ultrices arcu, vitae lacinia tellus tempor sit amet. Nulla non tincidunt dolor. In eget luctus sem, sed elementum ligula. Proin elementum adipiscing sem, sit amet ultricies nisl tincidunt eu. Ut lobortis dui quam, et scelerisque erat ultrices sit amet. Sed libero sem, mollis quis euismod quis, suscipit ac justo.

%% Sottosezione
\subsection{Sottosezione}
Donec cursus tortor eget sem ornare imperdiet. Ut vel orci non ipsum condimentum laoreet vitae ut sapien. Aenean metus mi, vehicula quis turpis nec, porttitor blandit dui. Nullam sed sollicitudin quam. Fusce nisl ante, commodo eget lacus ac, mollis ullamcorper neque. Quisque faucibus dictum nisl, dignissim fermentum sapien fringilla vel. Proin dui velit, molestie sit amet sapien et, pellentesque tristique purus. Curabitur ac quam ac diam varius bibendum.\part{Introduction}

%% Capitolo
\chapter{Problem: implementing advance IDE functionalities for the Jolie language to be added in Microsoft VSCode and similar}
In hac habitasse platea dictumst. Vestibulum consectetur dictum pellentesque. Suspendisse nunc neque, commodo ac imperdiet nec, sollicitudin vitae libero. Donec bibendum vel nunc vitae pharetra. In vel volutpat odio, et interdum dui. Duis mauris ligula, congue eget molestie at, tincidunt nec diam. Nam vitae eros nec arcu suscipit vehicula. Aliquam consectetur imperdiet elit, eget pretium arcu fringilla at. Maecenas \cite{Knu86} sed libero pulvinar, mattis tortor vel, fermentum enim.

%% Sezione
\section{Sezione}
Donec pulvinar neque non lectus vulputate pellentesque. Quisque rutrum arcu velit, in feugiat sapien posuere vel. Praesent metus orci, aliquam ac cursus eget, fermentum a nisl. Etiam eu augue lacus. Nam nisi sapien, mattis sed vehicula non, pellentesque at quam. Sed euismod, dolor nec commodo lobortis, erat erat ultricies eros, bibendum dictum nulla felis in dui. Nulla blandit ultrices arcu, vitae lacinia tellus tempor sit amet. Nulla non tincidunt dolor. In eget luctus sem, sed elementum ligula. Proin elementum adipiscing sem, sit amet ultricies nisl tincidunt eu. Ut lobortis dui quam, et scelerisque erat ultrices sit amet. Sed libero sem, mollis quis euismod quis, suscipit ac justo.

%% Sottosezione
\subsection{Sottosezione}
Donec cursus tortor eget sem ornare imperdiet. Ut vel orci non ipsum condimentum laoreet vitae ut sapien. Aenean metus mi, vehicula quis turpis nec, porttitor blandit dui. Nullam sed sollicitudin quam. Fusce nisl ante, commodo eget lacus ac, mollis ullamcorper neque. Quisque faucibus dictum nisl, dignissim fermentum sapien fringilla vel. Proin dui velit, molestie sit amet sapien et, pellentesque tristique purus. Curabitur ac quam ac diam varius bibendum.

\chapter{Jolie and the Language Server Protocol}
\section{Jolie: Java Orchestration Language Interpreter Engine}
\section{Language Server Protocol}

\part{Project}

\chapter{Analysis of the problem}
\section{Functional Requirments}
\subsection{Complying with the Language Server Protocol specifications}
\subsection{Language Server Protocol features}
\section{Non-functional requirments}
\subsection{Distributed}
\subsection{Modular/scalable}

\chapter{Design of the system}
\section{Arichitecture}
\subsection{Modules, interfaces, communication}
\section{What needs to be implemented}

\chapter{Implementation}
\section{Choosing a language: Jolie (why)}
\section{Implementation details}

\chapter{Validation}
\section{Met requirments (screeshots)}
\section{Unmet requirments}

\chapter{Conclusions: review, what needs to be done}

%% Fine dei capitoli normali, inizio dei capitoli-appendice (opzionali)
\appendix

\part{Appendici}

\chapter{Altro capitolo}
Sed purus libero, vestibulum ut nibh vitae, mollis ultricies augue. Pellentesque velit libero, tempor sed pulvinar non, fermentum eu leo. Duis posuere eleifend nulla eget sagittis. Nam laoreet accumsan rutrum. Interdum et malesuada fames ac ante ipsum primis in faucibus. Curabitur eget libero quis leo porttitor vehicula eget nec odio. Proin euismod interdum ligula non ultricies. Maecenas sit amet accumsan sapien.

%% Parte conclusiva del documento; tipicamente per riassunto, bibliografia e/o indice analitico.
\backmatter

%% Riassunto (opzionale)
\summary
Maecenas tempor elit sed arcu commodo, dapibus sagittis leo egestas. Praesent at ultrices urna. Integer et nibh in augue mollis facilisis sit amet eget magna. Fusce at porttitor sapien. Phasellus imperdiet, felis et molestie vulputate, mauris sapien tincidunt justo, in lacinia velit nisi nec ipsum. Duis elementum pharetra lorem, ut pellentesque nulla congue et. Sed eu venenatis tellus, pharetra cursus felis. Sed et luctus nunc. Aenean commodo, neque a aliquam bibendum, mauris augue fringilla justo, et scelerisque odio mi sit amet diam. Nulla at placerat nibh, nec rutrum urna. Donec ut egestas magna. Aliquam erat volutpat. Phasellus vestibulum justo sed purus mattis, vitae lacinia magna viverra. Nulla rutrum diam dui, vel semper mi mattis ac. Vestibulum ante ipsum primis in faucibus orci luctus et ultrices posuere cubilia Curae; Donec id vestibulum lectus, eget tristique est.

%% Bibliografia (opzionale)
\bibliographystyle{plain_\languagename}%% Carica l'omonimo file .bst, dove \languagename � la lingua attiva.
%% Nel caso in cui si usi un file .bib (consigliato)
\bibliography{thud}
%% Nel caso di bibliografia manuale, usare l'environment thebibliography.

%% Per l'indice analitico, usare il pacchetto makeidx (o analogo).

\end{document}

--- Istruzioni per l'aggiunta di nuove lingue ---
Per ogni nuova lingua utilizzata aggiungere nel preambolo il seguente spezzone:
    \addto\captionsitalian{%
        \def\abstractname{Sommario}%
        \def\acknowledgementsname{Ringraziamenti}%
        \def\authorcontactsname{Contatti dell'autore}%
        \def\candidatename{Candidato}%
        \def\chairname{Direttore}%
        \def\conclusionsname{Conclusioni}%
        \def\cosupervisorname{Co-relatore}%
        \def\cosupervisorsname{Co-relatori}%
        \def\cyclename{Ciclo}%
        \def\datename{Anno accademico}%
        \def\indexname{Indice analitico}%
        \def\institutecontactsname{Contatti dell'Istituto}%
        \def\introductionname{Introduzione}%
        \def\prefacename{Prefazione}%
        \def\reviewername{Controrelatore}%
        \def\reviewersname{Controrelatori}%
        %% Anno accademico
        \def\shortdatename{A.A.}%
        \def\summaryname{Riassunto}%
        \def\supervisorname{Relatore}%
        \def\supervisorsname{Relatori}%
        \def\thesisname{Tesi di \expandafter\ifcase\csname thud@target\endcsname Laurea\or Laurea Magistrale\or Dottorato\fi}%
        \def\tutorname{Tutor aziendale%
        \def\tutorsname{Tutor aziendali}%
    }
sostituendo a "italian" (nella 1a riga) il nome della lingua e traducendo le varie voci.
